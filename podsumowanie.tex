\chapter{Podsumowanie}

Celem pracy było stworzenie rozproszonego algorytmu weryfikacji modelowej na podstawie własności logicznych czasu liniowego LTL dla języka Alvis z wykorzystaniem frameworka Spring.
Wykonany został system, na który składa się kilka aplikacji komunikujących się za pomocą sieci.

Rozwiązanie to pozwala na weryfikację modelową w skalowalny sposób.
Wszystko dzięki bezstanowym serwisom oraz bazie NoSQL.
Wdrożenie na platformie chmurowej umożliwiłoby sprawdzanie wielu formuł LTL jednocześnie.
Wydajność byłaby zależna do liczby uruchomionych instancji.
Ponadto algorytm działa równolegle dla grup stanów wychodzących z jednego węzła, aby wykorzystać potencjał wielu rdzeni procesora i przyspieszyć długość weryfikacji pojedynczej formuły.

Sposób implementacji pozwala też na krokowe weryfikowanie.
To aplikacja eksplorująca stany nadaje tempo i kontroluje całość.
Umożliwia to nawet pauzowanie procesu, a dzięki trwałemu zapisowi w bazie danych - kontynuację w dowolnym momencie.

Kolejną cechą jest działanie w locie.
Pozwala na wczesne zakończenie algorytmu, czyli zanim przeszukana zostanie cała przestrzeń stanów.
Istnieją pewne warunki, które należy spełnić, aby do tego doszło.
Jednak sam ten fakt daje możliwość kontroli zmian w modelu z dużym prawdopodobieństwem szybkiego wykrycia problemów.
Równoległe uruchomienie wielu formuł pozwoli odnaleźć w locie znaczną część nowych błędów.

Aby mieć pewność, że system spełnia wszystkie założenia, trzeba oczywiście dokonać sprawdzenia wszystkich jego stanów.
Nie można orzec spełnienia formuły, jeśli proces zakończy się wcześniej.
Zwykle wymaga to długotrwałych obliczeń, ale dzięki implementacji, która umożliwia wdrożenie w środowisku chmurowym (z cechami opisanymi powyżej), czas ten ulega skróceniu.


\section{Propozycje rozwoju}

Podobnie jak w większości systemów tutaj także jest miejsce na usprawnienia.
Odnaleźć można kilka obszarów, które można rozwinąć.

Jeden z nich to sposób komunikacji pomiędzy aplikacjami GB i LV.
Interfejs aplikacji weryfikującej opiera się na protokole HTTP z wykorzystaniem formatu JSON.
Do wydajniejszego zakodowania danych użyć można np. Protocol Buffers, FlatBuffers, czy Apache Thrift.
Co więcej - zmiana modelu integracji na przesyłanie komunikatów pozwoliłaby na więcej równoległej pracy klienta i serwera.

Kolejny punkt na usprawnienie to baza danych.
Obecnie dwie tabele (przechowujące stany i tranzycje) używane są przez dwie aplikacje.
Jedynie GB dokonuje w nich modyfikacji, jednak potrzebne dane mogłyby być transportowane razem z komunikatem.
Pozbycie się integracji przez bazę danych zmniejszyłoby wzajemne zależności w systemie.

Ostatnim dostrzeżonym miejscem do ewentualnych zmian jest tekstowa reprezentacja formuły.
Wykorzystany język Kotlin świetnie sobie radzi z tym zadaniem, zapewnia pełną elastyczność, ale pozwala też na zbyt wiele.
Można za jego pomocą wykonać kod na serwerze LV, więc zastosowanie własnego języka opisu wraz z parserem byłoby bezpieczniejsze (lecz zapewne bardziej ograniczające).
