\chapter{Rozproszona implementacja}

Opisany algorytm dostosowany został do środowiska rozproszonego.
Taka implementacja pozwala uzyskać wysoką wydajność całego systemu, który staje się skalowalny.
Osiąga się to poprzez horyzontalne replikowanie aplikacji, która obsługuje weryfikację modelową.
Uruchomione zostaje wiele instancji tego samego serwisu.

Dzięki jego bezstanowości (żadnych danych sesyjnych, które przechowywane są w pamięci pomiędzy zapytaniami do serwera) nie ma znaczenia, która replika obsłuży żądanie.
Każda wykona je identycznie, a cały stan odczytywany i umieszczany jest z wykorzystaniem bazy danych.
W środowisku chmurowym system może być dynamicznie dostosowywany do aktualnych potrzeb, aby zapewnić szybkość, ale także nie marnować zasobów.

Poniżej przedstawiony zostanie opis zrealizowanego systemu.


\section{Graph Builder}

GB inicjuje cały algorytm oraz działa jako dostawca stanów dla weryfikatora (uzyskuje je od GP).
Najpierw wybierana jest formuła LTL, która zostanie sprawdzona (szczegóły jej specyfikacji w sekcji \ref{sec:LTL_formula_spec}).
Następnie zaczyna się już główna część algorytmu.
Pobierane są stany początkowe, po czym przekazywane są LV oraz heurystyce przeszukującej graf.

Każda weryfikacja posiada swój unikalny identyfikator, aby móc prowadzić wiele z nich równolegle na tym samym grafie stanów.
Nadawanie go następuje podczas wysyłania pierwszych stanów do LV.
Używa się go później w każdej komunikacji z weryfikatorem.

Wykorzystywane jest przeszukiwanie wszerz (BFS) zgodnie z fazą \textit{initialise} (algorytm \ref{alg:initialise}).
Kolejne stany uzyskiwane od GP zostają zapisywane w bazie danych, a także sprawdzane przez LV, który zwraca informacje o statusie weryfikacji.
Może ona zakończyć się szybciej (działanie w locie) w przypadku formuły, która nie jest spełniana przez graf.
W przeciwnym wypadku pobierane są kolejne stany aż do pełnej eksploracji.

Po odwiedzeniu wszystkich węzłów w grafie następuje wykonanie weryfikacji odpowiadającej liniom 4-8 w algorytmie \ref{alg:detectAcceptingCycle} przez LV.
Po tym etapie jednoznacznie wiadomo, czy formuła została spełniona.
Linie odwołujące się do \textit{verificationService} sprowadzają się wykonania zapytań do LV.

Kod prezentujący funkcjonalność realizowaną przez GB znajduje się w listingu \ref{lst:GB_pseudocode}.

\begin{minipage}{\linewidth}
\begin{lstlisting}[caption={Kod algorytmu realizowanego przez GB w języku Java.},captionpos=b,label={lst:GB_pseudocode},language=Java]
public VerificationState checkModel() {
  Collection<SystemState> states = initialStates.initialStates();
  List<StoredSystemState> saved = store.save(states).getNewlyStoredStates();
  heuristic.newStates(saved);
  VerificationId verificationId = verificationService.init(saved);
  while (heuristic.hasMoreToExplore()) {
    if (heuristic.shouldVisitState()) {
      StoredSystemState toVisit = heuristic.nextStateToVisit();
      Collection<Transition> transitions = description.allTransitionFromState(
          toVisit.getSystemState());
      heuristic.newTransitions(toVisit, transitions);
    }
    if (heuristic.shouldExploreTransition()) {
      Pair<StoredSystemState, Transition> toExplore = 
          heuristic.nextTransitionToExplore();
      Collection<SystemState> reachableStates = description.allReachableStates(
          toExplore.getFirst().getSystemState(), toExplore.getSecond());
      StoredStatesResult result = store.save(reachableStates);
      List<StoredSystemState> storedOnlyNewStates = result.getNewlyStoredStates();
      if (!storedOnlyNewStates.isEmpty()) {
        store.save(toExplore.getFirst(),
            storedOnlyNewStates, toExplore.getSecond());
        heuristic.newStates(storedOnlyNewStates);
      }
      List<StoredSystemState> storedOnlyOldStates = result.getAlreadyStoredStates();
      if (!storedOnlyOldStates.isEmpty()) {
        store.save(toExplore.getFirst(), storedOnlyOldStates,
            toExplore.getSecond());
      }
      List<StoredSystemState> allReachableStates =
          new ArrayList<>(storedOnlyNewStates);
      allReachableStates.addAll(storedOnlyOldStates);
      if (!allReachableStates.isEmpty()) {
        VerificationResult verificationResult = verificationService.verifyNewStates(
            verificationId, toExplore, allReachableStates);
        if (verificationResult.getVerificationState() == FORMULA_UNSATISFIED) {
          return FORMULA_UNSATISFIED;
        }
      }
    }
  }
  VerificationResult finishResult = verificationService.finish(verificationId);
  return finishResult.getVerificationState();
}
\end{lstlisting}
\end{minipage}


\section{Specyfikacja formuły LTL} \label{sec:LTL_formula_spec}

Obiektowa reprezentacja formuły LTL nie wystarcza, aby móc je tworzyć.
Potrzebny jest sposób umożliwiający zapis tekstem.
Dzięki temu można przygotowywać zestawy formuł, jakie powinien spełniać system, po czym wykorzystać je jako parametr dla weryfikatora.
Konieczne było więc ich parsowanie.

Spełnić należało dwa wymagania.
Pierwszym była obsługa spójników zdaniowych i operatorów temporalnych, jakie oferuje LTL.
Drugie to pozostawienie elastyczności, co do samych warunków, tak aby móc je dowolnie komponować i łączyć.
Wspierać należało porównywanie dowolnych wartości modelu na różne sposoby, np. przez tę samą wartość, wyrażenia regularne, czy relację porządku.

Zamiast tworzenia własnego języka, wykorzystany został Kotlin poprzez jego silnik skryptowy.
Pozwala na nieograniczoną elastyczność w tworzeniu formuł.
Kod wykonywany jest locie i zamieniany na reprezentację obiektową.
Jego składnia pozwala na elegancką notację z użyciem operatorów infiksowych.

Obiekt formuły opisany został w sekcji \ref{sec:ltl2buchi}.
Do porównywania wartości wykorzystana została biblioteka \textit{Hamcrest} (\url{http://hamcrest.org}).
Oferuje ona bogaty zestaw komparatorów, możliwość ich łączenia i testowania na dowolnym obiekcie systemu.

Przykładową formułę prezentuje listing \ref{lst:kotlin_formulas}.

\begin{minipage}{\linewidth}
\begin{lstlisting}[caption={Przykładowa formuła LTL w języku Kotlin.},captionpos=b,label={lst:kotlin_formulas}]

val bfsPcEq1 = proposition(hasProperty<SystemState>("bfsHeuristicState", 
        hasProperty<Int>("pc", equalTo(1))))
val bfsPcGt1 = proposition(hasProperty<SystemState>("bfsHeuristicState",
        hasProperty<Int>("pc", greaterThan(1))))
val formula = globally(bfsPcEq1 implies finally(bfsPcGt1))

\end{lstlisting}
\end{minipage}


\section{LTL to Büchi} \label{sec:ltl2buchi}

Do obiektowej reprezentacji formuły wykorzystana została biblioteka \textit{ltl2buchi} \\(\url{https://github.com/fraimondi/ltl2buchi}).
Pozwala ona na ich tworzenie z dowolnymi twierdzeniami (używa klas generycznych).

Jej główną funkcjonalnością jest ich konwersja do grafów reprezentujących uogólnione automaty Büchiego.
Te są redukowane, degeneralizowane i znów redukowane.
Brakuje jednak zautomatyzowanej możliwości poruszania się po tym automacie.

Biblioteka wymagała zaadaptowania do potrzeb projektu.
Wyodrębniona została funkcja konwertująca formułę do grafu (listing \ref{lst:formula_to_graph}).

\begin{minipage}{\linewidth}
\begin{lstlisting}[caption={Funkcja konwertująca formułę LTL do grafu.},captionpos=b,label={lst:formula_to_graph},language=Java]
  <T> Graph<T> ltlToBuchiAutomaton(Formula<T> formula) {
    val optimizedFormula = new Rewriter<>(formula).rewrite();
    Graph<T> gba = Translator.translate(optimizedFormula);
    Graph<T> reducedGba = SuperSetReduction.reduce(gba);
    Graph<T> ba = Degeneralize.degeneralize(reducedGba);
    Graph<T> sccReducedBa = SCCReduction.reduce(ba);
    Graph<T> simplifiedBa = Simplify.simplify(sccReducedBa);
    Graph<T> sfsReducedBa = SFSReduction.reduce(simplifiedBa);
    Formula.resetStatic();
    return sfsReducedBa;
  }

\end{lstlisting}
\end{minipage}

Drugim krokiem było stworzenie funkcji do budowania formuł, które wykorzystają potencjał Kotlina.
Każdy operator dwuargumentowy otrzymał wersję infiksową.
Kilka funkcji prezentuje listing \ref{lst:ltl_factory_extensions}.

\begin{minipage}{\linewidth}
\begin{lstlisting}[caption={Przykładowe funkcje do budowania formuł w języku Kotlin.},captionpos=b,label={lst:ltl_factory_extensions}]

  infix fun <T> Formula<Matcher<T>>.and(right: Formula<Matcher<T>>)
      : Formula<Matcher<T>> = LtlFactory.And(this, right)

  infix fun <T> Formula<Matcher<T>>.implies(right: Formula<Matcher<T>>)
      : Formula<Matcher<T>> = LtlFactory.Implies(this, right)

  infix fun <T> Formula<Matcher<T>>.or(right: Formula<Matcher<T>>)
      : Formula<Matcher<T>> = LtlFactory.Or(this, right)

  fun <T> next(formula: Formula<Matcher<T>>): Formula<Matcher<T>>
      = LtlFactory.Next(formula)

  fun <T> globally(formula: Formula<Matcher<T>>): Formula<Matcher<T>>
      = LtlFactory.Globally(formula)

\end{lstlisting}
\end{minipage}

Najważniejsza część integracji to implementacja klasy automatu Büchiego, która umożliwi przechodzenie pomiędzy węzłami na podstawie kolejnego stanu systemu.
Stworzony został algorytm, który wybiera krawędź, którą należy podążyć.
Szukana jest taka, aby spełnić wszystkie stojące na jej straży wyrażenia.
W tym celu eksploruje się wewnętrzną strukturę grafu, sprawdza listę wyrażeń oraz ich ewentualną negację.
Następnie zwrócony zostaje nowy automat, którego aktualny węzeł, to ten znaleziony.

Klasa pozwala także na sprawdzenie, czy aktualnie znajduje się w stanie akceptującym.
Mieści się tutaj również funkcja, która przyjmuje formułę, neguje ją oraz buduje automat Büchiego (zgodnie z listingiem \ref{lst:formula_to_graph}).
Kod prezentujący powyższą funkcjonalność znajduje się w listingu \ref{lst:ltl_automaton}.

\begin{minipage}{\linewidth}
\begin{lstlisting}[caption={Implementacja automatu Büchiego.},captionpos=b,label={lst:ltl_automaton},language=Java]
public class BuchiAutomaton<T> {
  private final Graph<Matcher<T>> buchiAutomaton;
  private final Node<Matcher<T>> currentNode;

  /* [...] */

  public BuchiAutomaton<T> nextState(T targetSystemState) {
    val outgoingEdges = this.currentNode.getOutgoingEdges();
    if (outgoingEdges.isEmpty()) {
      throw new IllegalStateException("No more edges to follow");
    }

    Node<Matcher<T>> nextNode = outgoingEdges.stream()
        .sorted(noGuardEdgeToTheEndComparator)
        .filter(edge -> allGuardsMatch(targetSystemState, edge))
        .findFirst()
        .map(Edge::getNext)
        .orElse(this.currentNode);

    return new BuchiAutomaton<>(this.buchiAutomaton, nextNode);
  }

  private boolean allGuardsMatch(Object system, Edge<Matcher<T>> edge) {
    return edge.getGuard().stream().allMatch(guard -> {
      val matches = guard.getAtom().matches(system);
      return guard.isNegated() != matches;
    });
  }

  public boolean isAccepting() {
    return this.currentNode.getAttributes().getString(ACCEPTING_KEY) != null;
  }

  public static <T> BuchiAutomaton<T> prepareAutomaton(
          Formula<Matcher<T>> formulaToCheck) {
    val negatedFormula = formulaToCheck.negate();
    return newBuchiAutomaton(negatedFormula);
  }
}

\end{lstlisting}
\end{minipage}

\newpage


\section{Weryfikator LTL}

Najważniejsza część algorytmu znajduje się module LV.
To bezstanowa aplikacja serwerowa, która wystawia interfejs przez protokół HTTP.
Możliwe są trzy typy zapytań:
\begin{itemize}
\item Inicjujące -- przyjmuje stany początkowe oraz formułę do sprawdzenia, po czym inicjuje algorytm. Ponadto zwraca \textit{VerificationId}, czyli identyfikator weryfikacji, który będzie potrzebny w kolejnych zapytaniach.
\item Odbierające kolejne stany systemu -- to element weryfikacji w locie, która może się zakończyć na tym etapie. W takim przypadku zwracana jest stosowana informacja do klienta. To jedyne zapytanie, które wykonuje się wielokrotnie dla jednej formuły (ilość zależna od wielkości grafu ze stanami systemu).
\item Finalizujące weryfikację -- wykorzystywane, kiedy cały graf stanów został zeksplorowany, a formuła jest spełniona lub zastosowana heurystyka nie znalazła kontrprzykładu. Wykonana zostaje pozostała część algorytmu.
\end{itemize}

Wszystkie elementy algorytmu, które należy współdzielić pomiędzy zapytaniami, przechowywane są w bazie danych.
Stworzone zostały serwisy, które obsługują tę funkcjonalność dla zbioru stanów poprzedzających cykl, wartości algorytmu MAP, czy samych automatów Büchiego.


\subsection{Inicjajcja weryfikacji modelowej}

Tutaj generowany jest identyfikator weryfikacji, a stany początkowe aplikowane są do automatu Büchiego.
Każdy z nich (po wykonaniu tranzycji) zostaje zapisany w bazie danych.
Ponadto stany systemu trafią do zbioru stanów poprzedzających cykl.
W listingu \ref{lst:initializeModelChecking} znajduje się kod implementujący tę funkcjonalność.

\begin{minipage}{\linewidth}
\begin{lstlisting}[caption={Kod obsługujący inicjację weryfikacji modelowej.},captionpos=b,label={lst:initializeModelChecking},language=Java]

public VerificationId initializeModelChecking(BuchiAutomaton<SystemState> automaton,
                                              List<String> initialStateHashes) {
  val verificationId = new VerificationId();
  initialStateHashes.stream()
      .map(systemStatesService::findStateByHash)
      .map(targetState -> {
        val nextState = automaton.nextState(targetState.getSystemState());
        return new StoredBuchiAutomaton(verificationId, nextState,
            targetState.getHash());
      })
      .peek(this::addToApproxSet)
      .forEach(buchiAutomatonRepository::save);
  return verificationId;
}

\end{lstlisting}
\end{minipage}


\subsection{Weryfikacja eksplorowanych stanów}

Wielokrotnie wykonywanym zapytaniem jest to, które weryfikuje nowe stany systemu.
Pozwala zakończyć sprawdzanie formuły LTL, jeśli znajdzie się akceptujący cykl dla jej zaprzeczenia.

Parametry wejściowe to lista nowych stanów, stan źródłowy, z którego one wychodzą oraz identyfikator weryfikacji.
Działanie zaczyna się od pobrania z bazy danych automatu Büchiego, który odpowiada źródłu.
Następnie przetwarzana jest cała lista nowych stanów w sposób równoległy, co pozwoli przyspieszyć wykonanie algorytmu.
Można było zastosować to podejście, ponieważ nie ma tu żadnej operacji, która modyfikuje wspólny zasób.
Wszystkie działania są rozłączne.

\begin{minipage}{\linewidth}
\begin{lstlisting}[caption={Implementacja heurystyki pozwalającej na weryfikację modelową w locie.},captionpos=b,label={lst:checkNewStates},language=Java]
public boolean checkNewStates(NewStates newStates, VerificationId verificationId) {
  String sourceStateHash = newStates.getSourceStateHash();
  val storedBuchiAutomaton = findSavedBuchiAutomaton(verificationId,
                                                     sourceStateHash);
  return newStates.getReachableStatesHashes().stream().parallel()
      .map(systemStatesService::findStateByHash)
      .map(targetState -> toNextStateAutomaton(verificationId, storedBuchiAutomaton,
                                               targetState))
      .peek(this::addToApproxSet)
      .map(buchiAutomatonRepository::save)
      .anyMatch(buchiAutomaton -> runMapAlgorithm(sourceStateHash, buchiAutomaton));
}

private boolean runMapAlgorithm(String sourceStateHash,
                                StoredBuchiAutomaton buchiAutomaton) {
  String targetStateHash = buchiAutomaton.getMatchingStateHash();
  VerificationId verificationId = buchiAutomaton.getVerificationId();
  String sourceStateMap =
      mapAlgorithmService.getMap(verificationId, sourceStateHash);
  if (buchiAutomaton.getBuchiAutomaton().isAccepting()) {
    if (targetStateHash.equals(sourceStateHash) ||
            targetStateHash.equals(sourceStateMap)) {
        log.info(buildAcceptingPath(buchiAutomaton));
        return true;
    }
    mapAlgorithmService.setMap(verificationId, targetStateHash,
        mapAlgorithmService.getMaxMap(targetStateHash, sourceStateMap));
  } else {
    mapAlgorithmService.setMap(verificationId, targetStateHash, sourceStateMap);
  }
  return false;
}

\end{lstlisting}
\end{minipage}

Dla każdego stanu uzyskuje się odpowiadający mu automat Büchiego (aplikując stan systemu do automatu źródłowego).
Kolejny krok to dodanie do zbioru stanów poprzedzających cykl, po czym następuje kluczowy moment, czyli zastosowanie heurystyki opartej na algorytmie MAP.
W przypadku wierzchołka akceptującego sprawdza się, czy nie został zamknięty cykl akceptujący.
Gdy do tego dojdzie, następuje zakończenie algorytmu, poinformowanie o tym klienta oraz wypisanie całej ścieżki od liścia do korzenia, która spowodowała niespełnienie zadanej formuły LTL. Całość działa zgodnie z opisem w sekcji \ref{heuristic}.

W pozostałych przypadkach (kiedy nie wykryto akceptującego cyklu) obliczane są kolejne wartości dla algorytmu MAP. Implementacja została zawarta w listingu \ref{lst:checkNewStates}.


\subsection{Finalizacja weryfikacji modelowej}

W przypadku, gdy nie dochodzi do wczesnego zakończenia algorytmu, a cała przestrzeń stanów została już wygenerowana, należy dokończyć sprawdzanie, zgodnie z OWCTY.
Jedyny argument wymagany od klienta to identyfikator weryfikacji.
Powala on na uzyskiwanie stanów w zbiorze poprzedników cyklu akceptującego.
Kolejne operacje na zbiorze wykonywane są w pętli do czasu, gdy przestaną zachodzić w nim zmiany lub okaże się pusty.
Brak elementów oznacza, że nie ma cyklu akceptującego. 
Przedstawia to listing \ref{lst:finishAlgorithm}.

Ze zbioru najpierw usuwane są stany nieakceptujące, po czym dodaje się do niego następniki pozostałych oraz ustala wartości liczby poprzedników (listing \ref{lst:eliminateNoAcceptingStates}).

Druga wykorzystywana funkcja to \textit{eliminateNoPredecessors} -- jej implementacja znajduje się w listing \ref{lst:eliminateNoPredecessorsStates}.
Jej zadanie polega na usunięciu ze zbioru wszystkich stanów, które nie mają poprzedników.
Co więcej, dla następników takich stanów, zmniejszana jest ich liczba poprzedników.
Jeśli spadnie ona do zera, całość powtarza się rekurencyjnie.

\begin{minipage}{\linewidth}
\begin{lstlisting}[caption={Kod obsługujący algorytm w przypadku, gdy nie zakończy się wcześniej.},captionpos=b,label={lst:finishAlgorithm},language=Java]
public boolean finishAlgorithm(VerificationId verificationId) {
  int oldSize = Integer.MAX_VALUE;
  while (true) {
    int approxSetSize = approximationSetService.size(verificationId);
    if (approxSetSize == oldSize || approxSetSize <= 0) {
      break;
    }
    oldSize = approxSetSize;
    eliminateNoAccepting(verificationId);
    eliminateNoPredecessors(verificationId);
  }
  return approximationSetService.size(verificationId) > 0;
}
\end{lstlisting}
\end{minipage}

\begin{minipage}{\linewidth}
\begin{lstlisting}[caption={Implementacja funkcji usuwającej stany nieakceptujące ze zbioru.},captionpos=b,label={lst:eliminateNoAcceptingStates},language=Java]
private void eliminateNoAccepting(String verificationId) {
  Queue<String> hashesQueue = new ArrayDeque<>();
  approximationSetService.streamAllStateHashes(verificationId)
      .forEach(stateHash -> {
        if (isAcceptingState(verificationId, stateHash)) {
          hashesQueue.offer(stateHash);
          approximationSetService.zeroPredecessorCount(verificationId, stateHash);
        } else { approximationSetService.remove(verificationId, stateHash); }
      });
  while (!hashesQueue.isEmpty()) {
    String sourceStateHash = hashesQueue.poll();
    findSuccessorStateHashes(sourceStateHash).forEach(stateHash -> {
          if (approximationSetService.contains(verificationId, stateHash)) {
            approximationSetService.incrementPredecessorCount(verificationId,
                                                              stateHash);
          } else {
            hashesQueue.offer(stateHash);
            approximationSetService.add(verificationId, stateHash);
            approximationSetService.zeroPredecessorCount(verificationId, stateHash);
          }
        });
  }
}
\end{lstlisting}
\end{minipage}
\begin{minipage}{\linewidth}
\begin{lstlisting}[caption={Implementacja funkcji usuwającej stany niemające poprzedników ze zbioru.},captionpos=b,label={lst:eliminateNoPredecessorsStates},language=Java]
private void eliminateNoPredecessors(String verificationId) {
  Queue<String> hashesQueue = new ArrayDeque<>();
  approximationSetService.streamAllStateHashes(verificationId)
      .filter(stateHash -> approximationSetService
          .getPredecessorCount(verificationId, stateHash) == 0)
      .forEach(hashesQueue::offer);
  while (!hashesQueue.isEmpty()) {
    String sourceStateHash = hashesQueue.poll();
    approximationSetService.remove(verificationId, sourceStateHash);
    findSuccessorStateHashes(sourceStateHash).forEach(stateHash -> {
          approximationSetService.decrementPredecessorCount(verificationId
                                                            stateHash);
          if (approximationSetService
              .getPredecessorCount(verificationId, stateHash) == 0) {
            hashesQueue.offer(stateHash);
          }
        });
  }
}
\end{lstlisting}
\end{minipage}


\section{Baza danych}

Funkcjonalność magazynu danych w systemie realizuje \textit{Apache Cassandra} (\url{https://cassandra.apache.org}).
To rozproszona baza NoSQL zaprojektowana do obsługi dużych zestawów danych.
Dostarcza wysoką dostępność bez ryzyka, że pojedyncza awaria uszkodzi system.
Oferuje wiele strategii replikacji, skalowalność i konfigurację poziomu spójności.

Cassandra została wykorzystana w projekcie m.in. do przechowywania eksplorowanych stanów, tranzycji, czy automatów Büchiego.

\noindent
Eksplorowane stany zapisywane są w tabeli o prostej strukturze.
Poza samą wartością znajduje się tutaj tylko identyfikator oraz wyliczony hasz, który służy do wyszukiwania:
\begin{itemize}
    \setlength\itemsep{-0.1cm}
    \item $ id: uuid $
    \item $ hash: text $
    \item $ systemstate: system\_state $
\end{itemize}
Tranzycje podobnie jak odwiedzane stany wykorzystywane są zarówno przez GB, jak i LV.
Każda z nich ma swą wartość oraz identyfikatory stanów, które łączy:
\begin{itemize}
    \setlength\itemsep{-0.1cm}
    \item $ id: uuid $
    \item $ fromsystemstateid: uuid $
    \item $ tosystemstateid: uuid $
    \item $ transition: transition $
\end{itemize}

Pozostałe tabele służą tylko i wyłącznie LV.
Przechowują dane, które muszą zostać zapamiętane pomiędzy kolejnymi zapytaniami klienta.
Najważniejsza z nich to zawierająca automaty Büchiego.
Serializowanie samego tego obiektu okazało się nietrywialne ze względu na skomplikowaną i cykliczną strukturę w \textit{ltl2buchi}.
Drugim problemem były obiekty porównujące z biblioteki \textit{Hamcrest}.
W związku z tym zapisywana w bazie jest formuła, z której można utworzyć automat oraz identyfikator aktualnego węzła, aby go ustawić.
Kolejna kolumna to \textit{verificationidwithstatehash}.
Będzie ona się powtarzać także w innych tabelach.
Pozwala powiązać rekord z konkretną weryfikacją oraz samym haszem stanu systemu, którego to dotyczy.
Wartość \textit{previousautomatonstateid} przechowuje identyfikator poprzedniego automatu, dzięki któremu można zawsze odnaleźć ścieżkę aż do korzenia (po znalezieniu akceptującego cyklu).
Całość tabeli prezentuje się następująco:
\begin{itemize}
    \setlength\itemsep{-0.1cm}
    \item $ id: uuid $
    \item $ previousautomatonstateid: uuid $
    \item $ currentnodeid: int $
    \item $ stringformula: text $
    \item $ verificationidwithstatehash: text $
\end{itemize}

W projekcie wykorzystane zostały jeszcze 3 tabele.
Potrzebne są, aby przechowywać dane, jakie do tej pory uzyskano przez algorytm.
Każda z nich składa się z identyfikatora, \textit{verificationidwithstatehash} (opisane powyżej) oraz dodatkowej kolumny specyficznej dla siebie.
Dla tabeli zawierającej zbiór stanów poprzedzających cykl akceptujący to hasz stanu systemu.
Kolejna zajmuje się ilością poprzedników dla każdego stanu, więc w jej skład wchodzi licznik.
Ostatnia obsługuje algorytm MAP, dlatego ją cechuje wartość MAP dla stanu.
