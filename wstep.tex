\chapter{Wstęp}

Weryfikacja modelowa to dziedzina umożliwiająca sprawdzenie systemu pod kątem specyfikacji.
Operacja taka jest zazwyczaj bardzo wymagająca pod kątem obliczeniowym, co skutkuje długimi czasami wykonania. Przeciwdziałać temu można na kilka sposobów, np. stosując uproszczenie modelu, czy wykorzystując wydajniejszy procesor. Kolejna możliwość to stworzenie skalowanego systemu rozproszonego i właśnie ta metoda zostanie rozważona.

Samą specyfikację wyrazić można na wiele sposobów. W pracy wykorzystana zostanie logika LTL (ang. \textit{linear-time temporal logic}).


\section{Cel pracy}

Celem pracy jest implementacja rozproszonego algorytmu weryfikacji modelowej w oparciu o własności logiczne czasu liniowego - LTL dla języka Alvis z wykorzystaniem frameworka Spring.


\section{Struktura pracy}

W następnym rozdziale pracy omówione zostanie zagadnienie weryfikacji modelowej w kontekście tematu pracy.
Rozdział trzeci zawiera ogólny opis architektury stworzonego systemu.
W czwartym rozdziale znajduja się szczegóły dotyczące zastosowanego algorytmu.
Piąty rozdział zawiera prezentację rozproszonej implementacji rozwiązania.
W ostatnim rozdziale umieszczone jest podsumowanie.
